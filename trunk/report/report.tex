\documentclass[a4paper]{article}
\usepackage{kotex}
\usepackage{palatino}
\usepackage[onehalfspacing]{setspace}
\usepackage{url}
\usepackage{calc}
\usepackage{multicol}
\addtolength{\textwidth}{30pt}

\begin{document}
\title{CS492 Distributed Systems \& Algorithms:\\ \textbf{Mr.CL}}
\author{20030767 최종욱, 20050145 김준기, 20060080 김민국}
\date{2009년 12월 24일}
\maketitle

\begin{multicols}{2}
\section{동기}
GPU를 일반 과학 연산에 활용하기 시작하면서 많은 성능 향상을 얻고 있지만 scalability의 한계점이 있었음

분산처리특강을 통해 Hadoop을 접하면서 map-reduce 방식이 그런 점을 극복하는 데 도움이 되리라 생각하고 GPU 가속을 결합시켜보고자 함

\section{알고리즘 설계}
\subsection{Hama 프로젝트}
TODO: Hama 프로젝트를 선택한 이유, 하지만 도입에 실패한 이유 설명 (HBase의 성능 제약 등)

TODO: 병목 지점이 Data I/O에 있음을 설명하고 다른 방식의 접근이 필요함을 설득

\subsection{Outer Product Algorithm}
TODO: outer product alogirthm 설명

\section{구현}
TODO: 최종적으로 Hadoop 위에 바로 코딩해서 올린 구조 설명

\section{성능 측정}
TODO: 성능 측정 계획 설명

\section{결론}
TODO: 부딪힌 문제점들 설명 (Hama와 HBase의 설치 중 삽질한 것, CUDA 드라이버 관련 삽질한 것, JCublas에서 double/float 삽질한 것 등)

TODO: 앞으로 개선할 점과 계속해서 연구한다면 어떤 것들을 해보고 싶다 등등

\end{multicols}
\end{document}
